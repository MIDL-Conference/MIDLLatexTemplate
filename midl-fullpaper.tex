\documentclass{midl} % Include author names

% The following packages will be automatically loaded:
% jmlr, amsmath, amssymb, natbib, graphicx, url, algorithm2e
% ifoddpage, relsize and probably more
% make sure they are installed with your latex distribution

\usepackage{mwe} % to get dummy images
\jmlrvolume{-- Under Review}
\jmlryear{2024}
\jmlrworkshop{Full Paper -- MIDL 2024 submission}
\editors{Under Review for MIDL 2024}

\title[Short Title]{Full Title of Article}

 % Use \Name{Author Name} to specify the name.
 % If the surname contains spaces, enclose the surname
 % in braces, e.g. \Name{John {Smith Jones}} similarly
 % if the name has a "von" part, e.g \Name{Jane {de Winter}}.
 % If the first letter in the forenames is a diacritic
 % enclose the diacritic in braces, e.g. \Name{{\'E}louise Smith}

 % Two authors with the same address
 % \midlauthor{\Name{Author Name1} \Email{abc@sample.edu}\and
 %  \Name{Author Name2} \Email{xyz@sample.edu}\\
 %  \addr Address}

 % Three or more authors with the same address:
 % \midlauthor{\Name{Author Name1} \Email{an1@sample.edu}\\
 %  \Name{Author Name2} \Email{an2@sample.edu}\\
 %  \Name{Author Name3} \Email{an3@sample.edu}\\
 %  \addr Address}


% Authors with different addresses:
% \midlauthor{\Name{Author Name1} \Email{abc@sample.edu}\\
% \addr Address 1
% \AND
% \Name{Author Name2} \Email{xyz@sample.edu}\\
% \addr Address 2
% }

%\footnotetext[1]{Contributed equally}

% More complicate cases, e.g. with dual affiliations and joint authorship
\midlauthor{\Name{Author Name1\midljointauthortext{Contributed equally}\nametag{$^{1,2}$}} \Email{abc@sample.edu}\\
\addr $^{1}$ Address 1 \\
\addr $^{2}$ Address 2 \AND
\Name{Author Name2\midlotherjointauthor\nametag{$^{1}$}} \Email{xyz@sample.edu}\\
\Name{Author Name3\nametag{$^{2}$}} \Email{alphabeta@example.edu}\\
\Name{Author Name4\midljointauthortext{Contributed equally}\nametag{$^{3}$}} \Email{uvw@foo.ac.uk}\\
\addr $^{3}$ Address 3 \AND
\Name{Author Name5\midlotherjointauthor\nametag{$^{4}$}} \Email{fgh@bar.com}\\
\addr $^{4}$ Address 4
}

\begin{document}

\maketitle

\begin{abstract}
This is a great paper and it has a concise abstract.
\end{abstract}

\begin{keywords}
List of keywords, comma separated.
\end{keywords}

\section{Introduction}

This is where the content of your paper goes.  Some random
notes\footnote{Random footnote are discouraged}:
\begin{itemize}
\item You should use \LaTeX \cite{Lamport:Book:1989}.
\item JMLR/PMLR uses natbib for references. For simplicity, here, \verb|\cite|  defaults to
  parenthetical citations, i.e. \verb|\citep|. You can of course also
  use \verb|\citet| for textual citations.
\item Eprints such as arXiv papers can of course be cited \cite{Hinton:arXiv:2015:Distilling}. We recomend using a \verb|@misc| bibtex entry for these as shown in the sample bibliography.
\item You should follow the guidelines provided by the conference.
\item Read through the JMLR template documentation for specific \LaTeX
  usage questions.
\item Note that the JMLR template provides many handy functionalities
such as \verb|\figureref| to refer to a figure,
e.g. \figureref{fig:example},  \verb|\tableref| to refer to a table,
e.g. \tableref{tab:example} and \verb|\equationref| to refer to an equation,
e.g. \equationref{eq:example}.
\end{itemize}

\begin{table}[htbp]
 % The first argument is the label.
 % The caption goes in the second argument, and the table contents
 % go in the third argument.
\floatconts
  {tab:example}%
  {\caption{An Example Table}}%
  {\begin{tabular}{ll}
  \bfseries Dataset & \bfseries Result\\
  Data1 & 0.12345\\
  Data2 & 0.67890\\
  Data3 & 0.54321\\
  Data4 & 0.09876
  \end{tabular}}
\end{table}

\begin{figure}[htbp]
 % Caption and label go in the first argument and the figure contents
 % go in the second argument
\floatconts
  {fig:example}
  {\caption{Example Image}}
  {\includegraphics[width=0.5\linewidth]{example-image}}
\end{figure}

\begin{algorithm2e}
\caption{Computing Net Activation}
\label{alg:net}
 % older versions of algorithm2e have \dontprintsemicolon instead
 % of the following:
 %\DontPrintSemicolon
 % older versions of algorithm2e have \linesnumbered instead of the
 % following:
 %\LinesNumbered
\KwIn{$x_1, \ldots, x_n, w_1, \ldots, w_n$}
\KwOut{$y$, the net activation}
$y\leftarrow 0$\;
\For{$i\leftarrow 1$ \KwTo $n$}{
  $y \leftarrow y + w_i*x_i$\;
}
\end{algorithm2e}

% Acknowledgments---Will not appear in anonymized version
\midlacknowledgments{We thank a bunch of people.}


\bibliography{midl-samplebibliography}


\appendix

\section{Proof of Theorem 1}

This is a boring technical proof of
\begin{equation}\label{eq:example}
\cos^2\theta + \sin^2\theta \equiv 1.
\end{equation}

\section{Proof of Theorem 2}

This is a complete version of a proof sketched in the main text.

\end{document}
